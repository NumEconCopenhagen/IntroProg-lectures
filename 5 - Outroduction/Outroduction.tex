% for resizing tables: https://tex.stackexchange.com/questions/38177/including-large-tables-in-a-beamer-frame

\documentclass[10pt,english,t,aspectratio=169]{beamer}
\usepackage{lmodern}
\usepackage[T1]{fontenc}
\usepackage[utf8]{inputenc}
%\usepackage[default]{lato} %font




\setlength{\parskip}{\smallskipamount}
\setlength{\parindent}{0pt}
\usepackage{babel}
\usepackage{amsmath}
\usepackage{amssymb}
\usepackage{bm} % like bold symbol but easier syntax, and says it is more careful 
%\numberwithin{equation}{section}
\usepackage{graphicx}
\usepackage{bbold}
\usepackage{makecell}


\usepackage{comment} % For longer notes
\ifx\hypersetup\undefined
  \AtBeginDocument{%
    \hypersetup{unicode=true}
  }
\else
  \hypersetup{unicode=true}
\fi
%\usepackage{breakurl}

\usepackage{tikzsymbols}% Smileys and stuf

\usepackage{booktabs} % Toprule midrule etc. in tables
\usepackage[flushleft]{threeparttable}
\usepackage{adjustbox} % Shrink stuff
%\usepackage{showframe} % Useful for debugging
\usepackage{dcolumn}  % For aligning decimals in tables

\usepackage{multirow}

\usepackage{pgfpages} % For when also showing notes 
% These slides also contain speaker notes. You can print just the slides,
% just the notes, or both, depending on the setting below. Comment out the want
% you want.
\setbeameroption{hide notes} % Only slides
%\setbeameroption{show only notes} % Only notes
%\setbeameroption{show notes on second screen=right} % Both

\usepackage{xcolor} % Colored math stuff using \mathcolor{blue}{Something}
\makeatletter
\def\mathcolor#1#{\@mathcolor{#1}}
\def\@mathcolor#1#2#3{%
  \protect\leavevmode
  \begingroup
    \color#1{#2}#3%
  \endgroup
}
\makeatother


\makeatletter
%%%%%%%%%%%%%%%%%%%%%%%%%%%%%% Textclass specific LaTeX commands.
% this default might be overridden by plain title style
\newcommand\makebeamertitle{\frame{\maketitle}}%
% (ERT) argument for the TOC
\AtBeginDocument{%
  \let\origtableofcontents=\tableofcontents
  \def\tableofcontents{\@ifnextchar[{\origtableofcontents}{\gobbletableofcontents}}
  \def\gobbletableofcontents#1{\origtableofcontents}
}

% For references
\usepackage{natbib}
\bibliographystyle{econ}

\usepackage{hyperref} %For link- references
\usepackage{changepage}  % For adjustwidth command



\usepackage{tikz} % Tikz inspiration by https://github.com/paulgp/beamer-tips/blob/master/slides.tex
\usetikzlibrary{positioning}

\usepackage{appendixnumberbeamer} % For stopping slidenumbering  using \appendix


\usetheme[progressbar=frametitle,block=fill]{metropolis}

% Set transparent overlay when using pause 
\setbeamercovered{transparent}

% code
\usepackage{listings} 

% margin
\setbeamersize{text margin right=0.5cm, text margin left=0.5cm }

% colors
\colorlet{DarkRed}{red!70!black}
\setbeamercolor{normal text}{fg=black}
\setbeamercolor{alerted text}{fg=DarkRed}
\setbeamercolor{progress bar}{fg=DarkRed}
\setbeamercolor{button}{bg=DarkRed}


% width of seperators
\makeatletter
\setlength{\metropolis@titleseparator@linewidth}{1pt}
\setlength{\metropolis@progressonsectionpage@linewidth}{1pt}
\setlength{\metropolis@progressinheadfoot@linewidth}{1pt}
\makeatother

% new alert block
\newlength\origleftmargini
\setlength\origleftmargini\leftmargini
\setbeamertemplate{itemize/enumerate body begin}{\setlength{\leftmargini}{4mm}}
\let\oldalertblock\alertblock
\let\oldendalertblock\endalertblock
\def\alertblock{\begingroup \setbeamertemplate{itemize/enumerate body begin}{\setlength{\leftmargini}{\origleftmargini}} \oldalertblock}
\def\endalertblock{\oldendalertblock \endgroup}
\setbeamertemplate{mini frame}{}
\setbeamertemplate{mini frame in current section}{}
\setbeamertemplate{mini frame in current subsection}{}
\setbeamercolor{section in head/foot}{fg=normal text.bg, bg=structure.fg}
\setbeamercolor{subsection in head/foot}{fg=normal text.bg, bg=structure.fg}


% Removes section slides
\begin{comment}
\AtBeginSection[]
{
}
\end{comment}


% Alternatively have slides with a table of contents at each new section
\begin{comment}
\AtBeginSection[]
{
    \begin{frame}[noframenumbering]
        \frametitle{Table of Contents}
        \tableofcontents[currentsection]
    \end{frame}
}
\end{comment}

% footer
\makeatletter
\setbeamertemplate{footline}{%
    \begin{beamercolorbox}[colsep=1.5pt]{upper separation line head}
    \end{beamercolorbox}
    \begin{beamercolorbox}{section in head/foot}
      \vskip1pt\insertsectionnavigationhorizontal{\paperwidth}{}{\hskip0pt plus1filll \insertframenumber{} / \inserttotalframenumber \hskip2pt}\vskip3pt% 
    \end{beamercolorbox}%
    \begin{beamercolorbox}[colsep=1.5pt]{lower separation line head}
    \end{beamercolorbox}
}
\makeatother

% toc
\setbeamertemplate{section in toc}{\hspace*{1em}\inserttocsectionnumber.~\inserttocsection\par}
\setbeamertemplate{subsection in toc}{\hspace*{2em}\inserttocsectionnumber.\inserttocsubsectionnumber.~\inserttocsubsection\par}

\makeatother
\title{Outroduction\vspace{-2mm}}
\subtitle{Introduction to Programming and Numerical Analysis \vspace{-4mm} } 
\author{Asker N. Christensen}
\date{Spring 2024}
%\today

%%%%%%%%%%%%%%%%%%%%%%%%%%%%%%%%%%%%%%%%%%%%%%%%%%%%%%%%%%%%%%%%%%%%%%%%%%%%%




\begin{document}



{
\setbeamertemplate{footline}{} 
\begin{frame}

\maketitle

\begin{tikzpicture}[overlay, remember picture]
\node[below left=0cm and 0cm of current page.north east] 
{\includegraphics[width=3 cm]{figs/ku_logo.png}};
\end{tikzpicture}
\end{frame}
}
\addtocounter{framenumber}{-1}



\section{Introduction}
\begin{frame}{Introduction}
    \begin{itemize}
        \item \textbf{We are almost at the end!}
        \item We hope that you have enjoyed the course, and will use programming in your further economics education/work.
        \item \textbf{\href{https://github.com/NumEconCopenhagen/IntroProg-lectures/blob/main/projects/ExamProject.pdf}{The final exam}} is a portfolio exam 
        \begin{enumerate}[-]
            \item The 3 projects you have created during the course.
            \item A new exam assignment, you'll have 48 hours to answer.
        \end{enumerate}
        
    \end{itemize}
\end{frame}

\begin{frame}{Table of contents}
    \tableofcontents[]
\end{frame}


\section{Course overview}
\begin{frame}{Course overview}
\begin{itemize}
    \item \textbf{Fundamentals:} Print, plot, optimize, simulate, structure, document, work-flow.  \par 
    \item \textbf{Working with data:} Fetch, combine, split-apply-combine, visualize.
    \item \textbf{Algorithms:} Pseudo code, algorithms, complexity, solve, optimization, symbolic.
    \item \textbf{Further perspectives:} Structural estimation, vectorization, parallelization, timing, numba.
\end{itemize}
    \vfill 
    While working with the material, you should hopefully also have gained the ability to \textbf{reformulate mathematical models into code.} \par 
    \textbf{All model notebooks} (ASAD, Labor supply, Consumer problem, Production economy,  Random numbers example, OLG, Ramsey) are relevant to revisit.
\end{frame}




\section{Exam project}
\begin{frame}{Problems}
\begin{itemize}
\item \textbf{Structure:} 3 problems with 3-6 sub-questions. Most problems are on solving and
simulating models and analyzing their implications graphically and
numerically. There can also be problems on working with data, and on algorithms. 
\item \textbf{Examples of a model problems}
\begin{enumerate}
\item Solve consumer or firm problems (with non-standard constraints)
\item Solve and simulate an AS-AD model
\item Solve for the Walras-equilibrium in an exchange economy
\item Solve an extended Solow model
\item Solve a two period dynamic optimization problem
\end{enumerate}
\item []$\Rightarrow$ similar to the problems in the problem sets
\item \textbf{Curriculum: }Lecture notebooks\textbf{ }(except notebooks and sections
marked with +)
\item \textbf{Packages: }No new packages are required, and using non-standard
packages are actively discouraged.
\end{itemize}
\end{frame}
%
\begin{frame}{Answering}
\begin{enumerate}
\item \textbf{Focus on answering the questions} - nothing more, nothing
less
\item Explain your \textbf{method in words }(or with an algorithm)
\item \textbf{Structure and comment your code!}
\item Explain your \textbf{results in words}
\item \textbf{Partial answers, attempts and considerations} are also \textbf{awarded
}(something on everything is better than a lot on a few questions)
\begin{enumerate}[-]
    \item If you think there is an error in your code, but you can't/don't have time to fix it, addressing this in words is a \textbf{huge} plus.
\end{enumerate}
\end{enumerate}
\textbf{Disclaimer: }Solving the full exam project in depth will be
hard. \par 
\textbf{Use of AI is allowed.} Guide to integrate with VS Code in the bottom of \underline{\href{https://sites.google.com/view/numeconcph-introprog/guides/installation}{this page}.}
\end{frame}
%
\begin{frame}{Hand-in}
\begin{itemize}
\item \textbf{You should hand-in a single zip-file named with your groupname
only.}
\item The zip-file should contain:
\begin{enumerate}
\item A general README.md for your portfolio
\item Your inaugural project (in the folder /inauguralproject)
\item Your data analysis project (in the folder /dataproject)
\item Your model analysis project (in the folder /modelproject)
\item Your exam project (in the folder /examproject)
\end{enumerate}
\item If you've uploaded everything to GitHub, you can download a zip of your repository from there (press the green 'code'-button)
\end{itemize}
\end{frame}
%







\section{Your to-do}
\begin{frame}{To-do before the exam}
    \begin{enumerate}
        \item[0.] Please answer the course evaluation you have gotten over mail.
        \item Model project \textbf{peer feedback} before \textbf{May 19}.
        \item \textbf{Polish up} your 3 projects so they are ready for the exam. 
        \item \textbf{Exam problems preparation.} \par
        Many ways to do that:
        \begin{itemize}
            \item Look at earlier exams to get a sense of the kind of questions you'll get
            \item Get an overview of the contents of the lecture and exercise notebooks. $\rightarrow$ Fill knowledge gaps.
            \item Go through the lectures that cover solving economic models \par 
            (ASAD, Labor supply, Consumer problem, Production economy,  Random numbers example, OLG, Ramsey). \par 
            Make sure you understand them, for example by changing them or rewriting the code to solve them yourself.
            \item Resolve the exercises.
            \item Try to solve earlier exams.
        \end{itemize}
    \end{enumerate}
\end{frame}



\begin{frame}{Questions}
\begin{itemize}
\item \textbf{Any questions now?}
\item \textbf{Online: }\href{https://github.com/NumEconCopenhagen/IntroProg-lectures/issues}{https://github.com/NumEconCopenhagen/IntroProg-lectures/issues}
\end{itemize}

\vfill
\textbf{Amped for more programming?} KU offers more programming courses: \underline{\href{https://sites.google.com/view/numeconcph/home}{https://sites.google.com/view/numeconcph/home}}
\end{frame}
%



\section{Forms questions}


\section{Exercises 6 and 7}

\section{2023 exam examples}





\end{document}